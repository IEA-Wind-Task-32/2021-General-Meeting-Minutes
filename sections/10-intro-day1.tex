\section{Day 1: Tuesday 16 November}

\bgroup
\begin{table}[!h]
 \centering
 % set up banded rows for the agenda and add lines to the columns
 \arrayrulecolor{Task32Blue2!15}
 \rowcolors{2}{Task32Blue2!5}{white}
 \begin{tabular}{@{}|p{0.125\columnwidth}|p{0.85\columnwidth}|@{}}
 \rowcolor{Task32Blue2} \textbf{Time} & \textbf{Activity} \\
 14:00 & Introductions\\
 15:00 & Panel session on “Wind Lidar for offshore wind energy applications”\par
  \begin{tableitemize}
        \item Julia Gottschall, Fraunhofer IWES
        \item Lina Poulsen, \O rsted
        \item Adria Miquel, Eolos Floating Lidar Solutions
        \item Neil Adams, Carbon Trust
    \end{tableitemize} \\
 16:00 & Networking session\\
 16:55 & Close \\
 \end{tabular}
 \label{tab:day1-agenda}
\end{table}
\egroup

Short breaks were taken between all sessions.

\subsection{Introductions}
The day started with a short introduction to the General Meeting by the Operating Agents. This and other presentations from this meeting are available through the Task 32 Zenodo repository at \href{https://doi.org/10.5281/zenodo.5718668}{10.5281/zenodo.5718668}.

Since 2019 Task 32 has been supporting the early stage researchers (ESRs) of the Innovation Training Network Marie Skłodowska-Curie Actions: Lidar Knowledge Europe (ITN LIKE) to build their professional networks and present their work. The ESRs focus on novel techniques and the application of wind lidar for wind energy and wind engineering.

The general coordinator of ITN LIKE, Charlotte Bay Hasager (DTU Wind Energy) presented a short overview of the ITN and the ESR's work. More information about LIKE is available at \href{https://www.msca-like.eu/}{www.msca-like.eu/}. 

\faFilePowerpointO ~The presentation about the ESRs and their research interests can be found at \href{https://doi.org/10.5281/zenodo.5726900
}{DOI: 10.5281/zenodo.5726900}.

All of the other participants in the meeting then introduced themselves. 

\clearpage