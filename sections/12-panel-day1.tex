\subsection{Panel discussion: “Wind Lidar for offshore wind energy applications”}

The panel included:
\begin{itemize}
        \item Julia Gottschall, Fraunhofer IWES
        \item Lina Poulsen, \O rsted
        \item Adria Miquel, Eolos Floating Lidar Solutions
        \item Neil Adams, Carbon Trust
    \end{itemize}

The panel explored the state of the art in using wind lidar for the deployment of wind energy offshore, and what R\&D are required. The discussion started with short presentations from each of the panelists about their perspectives on wind lidar for offshore wind energy applications, followed by a moderated discussion with audience questions. 75 people joined us.

Floating wind lidar have almost entirely replaced fixed-bottom met masts offshore in Europe, particularly for resource assessment. Scanning wind lidar are also used to provide data about the wind fields within operating plants, where they are placed on transformer platforms, turbine transition pieces, or the nacelle. Wind turbines are still increasing in size, with some models likely to reach 300-m tip heights in the next few years. This is within the measurement range of wind lidar, but is above the height of available validation facilities. New approaches to validation are therefore required to accurately quantify the uncertainty of wind lidar at these heights. However, it is also important to note that validation takes time; meeting climate protection goals will require speeding this up. Type certification (instead of unit certification) would help this process, but the reality is that wind lidar will seldom be the only source of data. So, it might be better to explore how to implement type certification for the entire system of wind lidar coupled with other technologies such as mesoscale modeling. As lidars replace traditional anemometers there is a growing need to use more data products derived from lidar measurements (especially turbulence intensity) to get more information about wind fields; this needs to be considered in the choice of validation methods. A lot of work is still very customer-specific rather than “off-the-shelf” and so care should be taken to develop standards and processes that allow different approaches to achieve the goals, rather than being restrictive.

\begin{taskactions}
``Accelerating offshore wind energy'' will be one of the four main themes in the relaunch of Task 32.
\end{taskactions}

\faFilePowerpointO ~ Find the presentations from the panelists in the meeting minutes at \href{https://doi.org/10.5281/zenodo.5718668}{DOI: 10.5281/zenodo.5718668}.
