\subsection{Plans for the future of Task 32}
Task 32 will reach the end of its current phase at the end of 2021. A proposal was made to the IEA Wind TCP Executive Committee (ExCo) in November 2021 to relaunch the Task. The proposal was led by Julia Gottschall from Fraunhofer IWES and David Schlipf. Julia Gottschall will replace Andrew Clifton from the University Stuttgart as the lead Operating Agent, representing the Task to the ExCo. David Schlipf will continue to be the second Operating Agent.

At the time of the Task 32 General Meeting no decision had been shared by the ExCo.

The session started with a handover from Andrew Clifton to Julia Gottschall. Julia then presented the plans for the relaunch of the Task. The plans are based on extensive consultation and have been refined in Workshops with the Task members during 2021.

The relaunched Task will focus on mitigating the challenges associated with the potential factor of five increase in the number of deployed wind lidar in the next 5 to 10 years. The relaunched Task's mission and vision have been updated accordingly:
\begin{itemize}
    \item \textbf{Mission}: We work together on research to make wind lidar the best and preferred wind measurement tool for wind energy applications
    \item \textbf{Vision}: Using wind lidar will be easy. It will bring advantages and opportunities that enable the deployment of wind energy
\end{itemize}

The Task will focus on 4 main themes. These are:
\begin{enumerate}
    \item \textbf{Universal inflow characterisation}. Working towards tools and methodologies to get and use the best information about inflow conditions to any wind turbine, anywhere.
    \item \textbf{Replacing met masts}. Creating guidelines for the selection and use of different types of wind lidar and software for site assessment.
    \item \textbf{Connecting wind lidar}. Helping users to improve measurements and extract value from their lidar(s) and data by making lidar data FAIR, enabling them to connect to an ecosystem of service providers
    \item \textbf{Accelerating offshore wind deployment}. Promoting wind lidar as a key enabling technology throughout the offshore wind project lifecycle.
\end{enumerate}

\faFilePowerpointO ~Find details of the relaunch plans at \href{https://doi.org/10.5281/zenodo.5163487
}{DOI: 10.5281/zenodo.5163487}.