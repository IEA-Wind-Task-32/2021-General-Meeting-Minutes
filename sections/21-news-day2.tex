\subsection{Community news}

\subsubsection{The new "Lidar data correction for sites in complex terrain (LoTar)" project}

Tobias Klaas-Witt from Fraunhofer IEE provided a short introduction to a new nationally-funded research project to create a open-source framework for modular lidar data processing. The project will build on the results of previous projects, including the Task 32 working group on ``wind lidar in complex terrain''. 

The project started in June 2021 and will run for 3 years. It is a joint project between Fraunhofer IEE and the  University of Stuttgart, who lead the project. More details can be found at \href{https://www.ifb.uni-stuttgart.de/forschung/windenergie/forschungsprojekte/LoTar/}{www.ifb.uni-stuttgart.de}.

\faFilePowerpointO ~ Find this presentation in the meeting minutes at \href{https://doi.org/10.5281/zenodo.5718668}{DOI: 10.5281/zenodo.5718668}.

\subsubsection{Update from the “wind lidar in complex terrain” working group} 

Alexander Stökl from Energiewerkstatt provided an update on this working group. Started in 2019, this group of around 10 industry and academic researchers has separately analysed data from several mountainous sites that are outside of the range of existing standards and compared them to data from meteorological towers. The results are being analysed at the moment and will be shared in a paper in 2022.

\faFilePowerpointO ~ Find this presentation in the meeting minutes at \href{https://doi.org/10.5281/zenodo.5718668}{DOI: 10.5281/zenodo.5718668}.

\subsubsection{Update from the “wind lidar in cold climate” working group} 

Marc Defossez from Nergica presented a short up date on this working group's progress. Set up in late 2020, the working group is a collaboration between Task 32 and Task 19 and will explore the opportunities and challenges for the use of wind lidar in cold climates. The group is currently identifying how it could have the most impact in this area.

Scripts to compare data from wind lidar with measurement masts were published in October 2021. They can be found in the Task 32 Github repository at \href{https://github.com/IEA-Wind-Task-32/cold-climate-data-comparison}{https://github.com/IEA-Wind-Task-32/cold-climate-data-comparison}

New participants are welcome. Please contact Marc directly for more information.

\faFilePowerpointO ~ Find this presentation in the meeting minutes at \href{https://doi.org/10.5281/zenodo.5718668}{DOI: 10.5281/zenodo.5718668}.

\subsubsection{The new round-robin on turbulence from forward-looking wind lidar}

Jakob von Eisenhart Rothe from DNV presented the plan for a round robin to compare the results from different methods of calculating turbulence intensity from a nacelle-mounted lidar. The round robin is a joint initiative between Task 32, OWC, DNV, ZX Lidars, and Leosphere.

\faFilePowerpointO ~ See details of the round robin at \href{https://doi.org/10.5281/zenodo.5713972}{DOI: 10.5281/zenodo.5713972}. The data for the round robin are available at \href{https://doi.org/10.5281/zenodo.5714038}{DOI: 10.5281/zenodo.5714038}.



